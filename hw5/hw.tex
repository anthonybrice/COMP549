\documentclass{abrice}

\title{Comp 549: Assignment 5}
\author{Anthony Brice}

\begin{document}
\maketitle

\section{Part 1}

\subsection{Put-That-There Demo, Bolt}

The application's ability to interpret voice and gesture commands is very
impressive, particularly its ability understand pronouns as referent to objects
given through gesture. That said, the interface seems restricted to a fairly
trivial domain.

\subsection{Samsung Gesture Recognition Technology TV Demo At CES 2013}

I found it funny that the reporter had to get out of view of the television's
camera in order for it to recognize the PR rep's hand. It seems that the
technology in the previous video was much more advanced in that regard.

Overall, this demonstration is impressive in that it presents a very practical
domain for the technology presented in the previous video. I'd love to replace
my television remote with voice and gesture commands, and I think it works
because the way in which one uses a television is only minimally interactive.

\subsection{Google's New Finger Control Technology}

I'm a little skeptical about practical applications of this form of
interaction. As Suchman notes, ``designers have long held the view that ideally
a device should be \ldots\ decipherable solely from information provided by the
device itself.'' I imagine with this technology it would be particularly
difficult to convey to a user the ways in which he or she can interact and even
the location within which one interacts with the device. Perhaps my fears will
be shown unfounded if Google can demonstrate that a small vocabulary of gestures
can provide rich interaction with a device, but the video itself didn't
show only real use cases, only target demonstrations.

\section{Part 2}

\subsection{From Turtles to Tangible Programming Bricks: Exploration in Physical
Language Design, McNerney}

While I found all the research described in the paper commendable, I was
especially interested in McNerney's application of functional programming
concepts in his Digital Construction Set. I agree that ``[p]rograms
using streams are naturally concise,'' and its always validating to hear a
prominent researcher say it.

I was initially dismayed that the researchers described in the paper seem to
have dismissed video games entirely from their purview, but considering the
paper was published in 2003 and covers research dating back 35 years I can't say
I blame them. \emph{Minecraft} wasn't released until 2009 and seems to
fit to a tee Papert's ``concept of a \emph{microworld}, a deliberately
simplified computational environment that allows students to explore ideas that
are not normally demonstrable in an average classroom setting.'' I would be very
interested to get McNerney's take on the game's efficacy in teaching children
programming through its integration of Boolean algebra

\end{document}
%%% Local Variables:
%%% mode: latex
%%% TeX-master: t
%%% End:
