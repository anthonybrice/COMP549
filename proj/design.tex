\documentclass{abrice}

\title{Comp 549: Project Design}
\author{Anthony Brice}
\date{\today\protect\\ \bigskip The Computer Algebra Tutor}
\setcounter{secnumdepth}{2}

\begin{document}
\maketitle

\section{Introduction}

This document describes the design and implementation of the a Computer Algebra
Tutor for use by high school- and college-age students. It references the
proposal document written at the outset of project development.

\section{System Design}

Informed by my research, it was important to me that the system minimize the
negative affects among users. I had to make clear to the user when mistake was
made, but I tried to be careful about not overstating the error. This also
informed my decision about whether a student should be able to continue a
problem after an error, equating subsequent expressions with the erroneous
one. I chose to disallow this since as proposed the primary goal of the project
was to reinforce algebraic concepts.

\section{Hardware/Software Design}

To facilitate the implementation and ease of deployment, I decided to develop
the project as a client/server application with the front-end running in a
standard web browser. This allowed me to develop a front-end that can run on
desktop or mobile devices with the same set of code. I would have liked the
backend to run on any device as well to keep the application discrete, but that
would require porting Seshat, the program that takes hand-drawn input and
converts it to \TeX, to various devices and accessing it as a library rather than
a binary as its designer intended.

The front-end was implemented in standard HTML5/CSS/ECMAScript with jQuery and
Bootstrap to ensure proper sizing of text and links across devices. I considered
Fitts's law when designing my inputs, but since the main mode of interaction on
mobile uses no pointer device and is not fullscreen on desktop, the law seemed
inapplicable. I ran into much trouble trying to resize the area for hand-drawn
input (the canvas) for smaller devices. The libraries I used were hardcoded for
an area of specific pixel width and height, and changing the dimensions caused
input to map improperly. I chose to present the canvas at the bottom of the page
with parsed TeX appearing directly above it so that the output would match how
students tend to attack math problems, a complex expression with subsequent
simplifications appearing below it.

The backend was implemented as a simple Node.js server hosting the front-end and
providing an API for parsing sketches with Seshat and for equating expressions
with the Wolfram Alpha API. Believing one to exist, I intended to use a MathML
parser for equating expressions. This proved not to be the case and implementing
one in the time allotted proved beyond my abilities. That said, the Wolfram
Alpha API proved very capable, but my use is unfortunately limited to 2000 calls
a month.



\end{document}
%%% Local Variables:
%%% mode: latex
%%% TeX-master: t
%%% End:
