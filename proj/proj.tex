\documentclass{abrice}

\title{Comp 549: Project Proposal}
\author{Anthony Brice}
\date{\today\protect\\ \bigskip The Computer Algebra Tutor}
\setcounter{secnumdepth}{2}
\begin{document}
\maketitle

\section{Introduction}

Many high school- and college-age students face considerable difficulty taking
their first serious mathematics courses due to the level of literacy required in
both basic algebra syntax and how to translate that for a standard algebraic
calculator. While less fundamental than the former, the latter skill is yet
critical because students often use their calculators as a means to verify their
intuition, and any mistranslation can easily lead them astray. The
oft-encountered algebra student insisting that the square of $-1$ is $-1$
attests to this fact.

This paper proposes an interactive system that would aid both areas of
instruction by allowing students to work out algebra problems naturally in any
desired syntax, providing feedback when a step has resulted in an unequal
expression. The system will accept input from any front-end that can parse math
expressions into some specified markup (e.g.\ MathML), display the expression
back to the reader in something human-readable (e.g.\ compiled \TeX), and on
subsequent iterations of use will inform the user if the next expression is
equivalent to the last. This proposal also includes two reference front-ends,
one for handwritten math expressions and one for a keyboard-oriented syntax.

\section{Requirements}

A complete system consists of a common back-end providing an API to one of a
number of front-ends differing on the way they accept input.

\subsection{Functional Requirements}

\begin{enumerate}
\item Back-end
  \begin{enumerate}
  \item Accept two mathematics expressions and return whether they are
    equivalent.
  \end{enumerate}
\item Front-end (Common)
  \begin{enumerate}
  \item Parse mathematics expressions into some computer-readable format.
  \item Display to the user the parsed expressions in a human-readable format.
  \item Display to the user if two subsequent expressions are inequivalent.
  \end{enumerate}
\item Front-end (Keyboard-oriented syntax)
  \begin{enumerate}
  \item Parse keyboard-written math expressions (i.e. like WolframAlpha syntax)
    into some computer-readable format
  \end{enumerate}
\item Front-end (Handwritten syntax)
  \begin{enumerate}
  \item Parse handwritten math expressions into some computer-readable format.
  \end{enumerate}
\end{enumerate}

\subsection{Performance Requirements}

Feedback to the user should be  less than 1 second for algebraic expressions of
up to 3 variables.

\subsection{Cost}

Any reasonable desktop computer (say 1000 USD) should be capable of running the
system.

\section{Implementation}

The back-end will be implemented to run on standard hardware. The front-ends
described will be implemented to run in a browser, but the proposal's
requirements give no reason such that one could not run natively.

\section{Project Plan}

Given the time left in the semester, I plan two weeks of 10 hours for the
front-end and 3 weeks of 10 hours for the front-ends.

\end{document}
%%% Local Variables:
%%% mode: latex
%%% TeX-master: t
%%% End:
